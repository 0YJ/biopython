\chapter{Sequence alignments}
\label{chapter:align}

\section{Alignment objects}
\label{sec:alignmentobject}

The \verb+aligner.align+ method returns \verb+Alignment+ objects, each representing one alignment between the two sequences.

%doctest
\begin{minted}{pycon}
>>> from Bio import Align
>>> aligner = Align.PairwiseAligner()
>>> target = "GAACT"
>>> query = "GAT"
>>> alignments = aligner.align(target, query)
>>> alignment = alignments[0]
>>> alignment  # doctest: +ELLIPSIS
<Alignment object (2 rows x 5 columns) at ...>
\end{minted}

Each alignment stores the alignment score:

%cont-doctest
\begin{minted}{pycon}
>>> alignment.score
3.0
\end{minted}
as well as pointers to the sequences that were aligned:

%cont-doctest
\begin{minted}{pycon}
>>> alignment.target
'GAACT'
>>> alignment.query
'GAT'
\end{minted}

Print the \verb+Alignment+ object to show the alignment explicitly:

%cont-doctest
\begin{minted}{pycon}
>>> print(alignment)
target            0 GAACT 5
                  0 ||--| 5
query             0 GA--T 3
<BLANKLINE>
\end{minted}

Internally, the alignment is stored in terms of the sequence coordinates:

%cont-doctest
\begin{minted}{pycon}
>>> alignment.coordinates
array([[0, 2, 4, 5],
       [0, 2, 2, 3]])
\end{minted}

Here, the two rows refer to the target and query sequence. These coordinates show that the alignment consists of the following three blocks:

\begin{itemize}
\item \verb+target[0:2]+ aligned to \verb+query[0:2]+;
\item \verb+target[2:4]+ aligned to a gap, since \verb+query[2:2]+ is an empty string;
\item \verb+target[4:5]+ aligned to \verb+query[2:3]+.
\end{itemize}

The number of aligned sequences is returned by \verb+len(alignment)+; this is
always 2 for a pairwise alignment:

%cont-doctest
\begin{minted}{pycon}
>>> len(alignment)
2
\end{minted}

The alignment length is defined as the number of columns in the alignment as
printed. This is equal to the sum of the number of matches, number of
mismatches, and the total length of gaps in the target and query:

%cont-doctest
\begin{minted}{pycon}
>>> alignment.length
5
\end{minted}

The \verb+shape+ property returns a tuple consisting of the length of the
alignment and the number of columns in the alignment as printed:

%cont-doctest
\begin{minted}{pycon}
>>> alignment.shape
(2, 5)
\end{minted}

For local alignments, sections that are not aligned are not included in the number of columns:
%cont-doctest
\begin{minted}{pycon}
>>> aligner.mode = "local"
>>> local_alignments = aligner.align("TGAACT", "GAC")
>>> local_alignment = local_alignments[0]
>>> print(local_alignment)
target            1 GAAC 5
                  0 ||-| 4
query             0 GA-C 3
<BLANKLINE>
>>> local_alignment.shape
(2, 4)
\end{minted}

Use the \verb+aligned+ property to find the start and end indices of subsequences in the target and query sequence that were aligned to each other.
Generally, if the alignment between target (t) and query (q) consists of $N$
chunks, you get a numpy array with dimensions $2 \times N \times 2$:

\begin{minted}{python}
(
    ((t_start1, t_end1), (t_start2, t_end2), ..., (t_startN, t_endN)),
    ((q_start1, q_end1), (q_start2, q_end2), ..., (q_startN, q_endN)),
)
\end{minted}

In the current example, \verb+alignment.aligned+ returns two tuples of length 2:

%cont-doctest
\begin{minted}{pycon}
>>> alignment.aligned
array([[[0, 2],
        [4, 5]],
<BLANKLINE>
       [[0, 2],
        [2, 3]]])
\end{minted}
while for the alternative alignment, two tuples of length 3 are returned:

%cont-doctest
\begin{minted}{pycon}
>>> alignment = alignments[1]
>>> print(alignment)
target            0 GAACT 5
                  0 |-|-| 5
query             0 G-A-T 3
<BLANKLINE>
>>> alignment.aligned
array([[[0, 1],
        [2, 3],
        [4, 5]],
<BLANKLINE>
       [[0, 1],
        [1, 2],
        [2, 3]]])
\end{minted}
Note that different alignments may have the same subsequences aligned to each other. In particular, this may occur if alignments differ from each other in terms of their gap placement only:

%cont-doctest
\begin{minted}{pycon}
>>> aligner.mode = "global"
>>> aligner.mismatch_score = -10
>>> alignments = aligner.align("AAACAAA", "AAAGAAA")
>>> len(alignments)
2
>>> print(alignments[0])
target            0 AAAC-AAA 7
                  0 |||--||| 8
query             0 AAA-GAAA 7
<BLANKLINE>
>>> alignments[0].aligned
array([[[0, 3],
        [4, 7]],
<BLANKLINE>
       [[0, 3],
        [4, 7]]])
>>> print(alignments[1])
target            0 AAA-CAAA 7
                  0 |||--||| 8
query             0 AAAG-AAA 7
<BLANKLINE>
>>> alignments[1].aligned
array([[[0, 3],
        [4, 7]],
<BLANKLINE>
       [[0, 3],
        [4, 7]]])
\end{minted}
The \verb+aligned+ property can be used to identify alignments that are identical to each other in terms of their aligned sequences.

The \verb+sort+ method sorts the alignment sequences. By default, sorting is done based on the \verb+id+ attribute of each sequence if available, or the sequence contents otherwise.
%cont-doctest
\begin{minted}{pycon}
>>> print(local_alignment)
target            1 GAAC 5
                  0 ||-| 4
query             0 GA-C 3
<BLANKLINE>
>>> local_alignment.sort()
>>> print(local_alignment)
target            0 GA-C 3
                  0 ||-| 4
query             1 GAAC 5
<BLANKLINE>
\end{minted}
Alternatively, you can supply a \verb+key+ function to determine the sort order. For example, you can sort the sequences by increasing GC content:
%cont-doctest
\begin{minted}{pycon}
>>> from Bio.SeqUtils import gc_fraction
>>> local_alignment.sort(key=gc_fraction)
>>> print(local_alignment)
target            1 GAAC 5
                  0 ||-| 4
query             0 GA-C 3
<BLANKLINE>
\end{minted}
The \verb+reverse+ argument lets you reverse the sort order to obtain the sequences in decreasing GC content:
%cont-doctest
\begin{minted}{pycon}
>>> local_alignment.sort(key=gc_fraction, reverse=True)
>>> print(local_alignment)
target            0 GA-C 3
                  0 ||-| 4
query             1 GAAC 5
<BLANKLINE>
\end{minted}

The \verb+frequencies+ method calculates how often each letter appears in each column of the alignment:
%cont-doctest
\begin{minted}{pycon}
>>> target = "AAAAAAAACCCCCCCCGGGGGGGGTTTTTTTT"
>>> query = "AAAAAAACCCTCCCCGGCCGGGGTTTAGTTT"
>>> aligner.mismatch_score = -1
>>> aligner.gap_score = -1
>>> alignments = aligner.align(target, query)
>>> len(alignments)
8
>>> print(alignments[0])
target            0 AAAAAAAACCCCCCCCGGGGGGGGTTTTTTTT 32
                  0 |||||||-|||.||||||..|||||||..||| 32
query             0 AAAAAAA-CCCTCCCCGGCCGGGGTTTAGTTT 31
<BLANKLINE>
>>> alignments[0].frequencies  # doctest: +NORMALIZE_WHITESPACE
{'A': array([2, 2, 2, 2, 2, 2, 2, 1, 0, 0, 0, 0, 0, 0, 0, 0, 0, 0, 0, 0, 0, 0, 0, 0, 0, 0, 0, 1, 0, 0, 0, 0]),
 'C': array([0, 0, 0, 0, 0, 0, 0, 0, 2, 2, 2, 1, 2, 2, 2, 2, 0, 0, 1, 1, 0, 0, 0, 0, 0, 0, 0, 0, 0, 0, 0, 0]),
 'G': array([0, 0, 0, 0, 0, 0, 0, 0, 0, 0, 0, 0, 0, 0, 0, 0, 2, 2, 1, 1, 2, 2, 2, 2, 0, 0, 0, 0, 1, 0, 0, 0]),
 'T': array([0, 0, 0, 0, 0, 0, 0, 0, 0, 0, 0, 1, 0, 0, 0, 0, 0, 0, 0, 0, 0, 0, 0, 0, 2, 2, 2, 1, 1, 2, 2, 2])}
\end{minted}
Use the \verb+substitutions+ method to find the number of substitutions between each pair of nucleotides:
%cont-doctest
\begin{minted}{pycon}
>>> m = alignments[0].substitutions
>>> print(m)
    A   C   G   T
A 7.0 0.0 0.0 0.0
C 0.0 7.0 0.0 1.0
G 0.0 2.0 6.0 0.0
T 1.0 0.0 1.0 6.0
<BLANKLINE>
\end{minted}

Note that the matrix is not symmetric: rows correspond to the target sequence, and columns to the query sequence.  For example, the number of G's in the target sequence that are aligned to a C in the query sequence is
%cont-doctest
\begin{minted}{pycon}
>>> m["G", "C"]
2.0
\end{minted}
and the number of C's in the query sequence tat are aligned to a T in the query sequence is
%cont-doctest
\begin{minted}{pycon}
>>> m["C", "G"]
0.0
\end{minted}
To get a symmetric matrix, use
%cont-doctest
\begin{minted}{pycon}
>>> m += m.transpose()
>>> m /= 2.0
>>> print(m)
    A   C   G   T
A 7.0 0.0 0.0 0.5
C 0.0 7.0 1.0 0.5
G 0.0 1.0 6.0 0.5
T 0.5 0.5 0.5 6.0
<BLANKLINE>
>>> m["G", "C"]
1.0
>>> m["C", "G"]
1.0
\end{minted}
The total number of substitutions between C's and G's in the alignment is 1.0 + 1.0 = 2.

The \verb+map+ method can be applied on a pairwise alignment \verb+alignment1+ to find the pairwise alignment of the query of \verb+alignment2+ to the target of \verb+alignment1+, where the target of \verb+alignment2+ and the query of \verb+alignment1+ are identical. A typical example is where \verb+alignment1+ is the pairwise alignment between a chromosome and a transcript, \verb+alignment2+ is the pairwise alignment between the transcript and a sequence (e.g., an RNA-seq read), and we want to find the alignment of the sequence to the chromosome:

%cont-doctest
\begin{minted}{pycon}
>>> aligner.mode = "local"
>>> aligner.open_gap_score = -1
>>> aligner.extend_gap_score = 0
>>> chromosome = "AAAAAAAACCCCCCCAAAAAAAAAAAGGGGGGAAAAAAAA"
>>> transcript = "CCCCCCCGGGGGG"
>>> alignments1 = aligner.align(chromosome, transcript)
>>> len(alignments1)
1
>>> alignment1 = alignments1[0]
>>> print(alignment1)
target            8 CCCCCCCAAAAAAAAAAAGGGGGG 32
                  0 |||||||-----------|||||| 24
query             0 CCCCCCC-----------GGGGGG 13
<BLANKLINE>
>>> sequence = "CCCCGGGG"
>>> alignments2 = aligner.align(transcript, sequence)
>>> len(alignments2)
1
>>> alignment2 = alignments2[0]
>>> print(alignment2)
target            3 CCCCGGGG 11
                  0 ||||||||  8
query             0 CCCCGGGG  8
<BLANKLINE>
>>> mapped_alignment = alignment1.map(alignment2)
>>> print(mapped_alignment)
target           11 CCCCAAAAAAAAAAAGGGG 30
                  0 ||||-----------|||| 19
query             0 CCCC-----------GGGG  8
<BLANKLINE>
>>> format(mapped_alignment, "psl")
'8\t0\t0\t0\t0\t0\t1\t11\t+\tquery\t8\t0\t8\ttarget\t40\t11\t30\t2\t4,4,\t0,4,\t11,26,\n'
\end{minted}

Mapping the alignment does not depend on the sequence contents. If we delete
the sequence contents, the same alignment is found in PSL format (though we
obviously lose the ability to print the sequence alignment):

%cont-doctest
\begin{minted}{pycon}
>>> from Bio.Seq import Seq
>>> alignment1.target = Seq(None, len(alignment1.target))
>>> alignment1.query = Seq(None, len(alignment1.query))
>>> alignment2.target = Seq(None, len(alignment2.target))
>>> alignment2.query = Seq(None, len(alignment2.query))
>>> mapped_alignment = alignment1.map(alignment2)
>>> format(mapped_alignment, "psl")
'8\t0\t0\t0\t0\t0\t1\t11\t+\tquery\t8\t0\t8\ttarget\t40\t11\t30\t2\t4,4,\t0,4,\t11,26,\n'
\end{minted}

\section{Slicing and indexing an alignment}

Slices of the form \verb+alignment[k, i:j]+, where \verb+k+ is an integer and \verb+i+ and \verb+j+ are integers or are absent, return a string showing the aligned sequence (including gaps) for the target (if \verb+k=0+) or the query (if \verb+k=1+) that includes only the columns \verb+i+ through \verb+j+ in the printed alignment.

To illustrate this, in the following example the printed alignment has 5 columns:

%cont-doctest
\begin{minted}{pycon}
>>> print(alignment)
target            0 GAACT 5
                  0 |-|-| 5
query             0 G-A-T 3
<BLANKLINE>
\end{minted}

To get the aligned sequence strings individually, use
%cont-doctest
\begin{minted}{pycon}
>>> alignment[0]
'GAACT'
>>> alignment[1]
'G-A-T'
>>> alignment[0, :]
'GAACT'
>>> alignment[1, :]
'G-A-T'
>>> alignment[0, 1:-1]
'AAC'
>>> alignment[1, 1:-1]
'-A-'
\end{minted}

Columns to be included can also be selected using an iterable over integers:
%cont-doctest
\begin{minted}{pycon}
>>> alignment[0, (1, 3, 4)]
'ACT'
>>> alignment[1, range(0, 5, 2)]
'GAT'
\end{minted}

To get specific columns in the alignment, use
%cont-doctest
\begin{minted}{pycon}
>>> alignment[:, 0]
'GG'
>>> alignment[:, 1]
'A-'
>>> alignment[:, 2]
'AA'
\end{minted}

Slices of the form \verb+alignment[:, i:j]+, where \verb+i+ and \verb+j+ are integers or are absent, return a new \verb+Alignment+ object that includes only the columns \verb+i+ through \verb+j+ in the printed alignment.

Extracting the first 4 columns for the example alignment above gives:
%cont-doctest
\begin{minted}{pycon}
>>> alignment[:, :4]  # doctest:+ELLIPSIS
<Alignment object (2 rows x 4 columns) at ...>
>>> print(alignment[:, :4])
target            0 GAAC 4
                  0 |-|- 4
query             0 G-A- 2
<BLANKLINE>
\end{minted}
Here, the final \verb+T+ nucleotides are still shown, but they are not aligned to each other. Note that \verb+alignment+ is a global alignment, but \verb+alignment[:, :4]+ is a local alignment.

Similarly, extracting the last 3 columns gives:
%cont-doctest
\begin{minted}{pycon}
>>> alignment[:, -3:]  # doctest:+ELLIPSIS
<Alignment object (2 rows x 3 columns) at ...>
>>> print(alignment[:, -3:])
target            2 ACT 5
                  0 |-| 3
query             1 A-T 3
<BLANKLINE>
\end{minted}
This is also now a local alignment, with the initial \verb+GA+ nucleotides in the target and \verb+G+ nucleotide in the query not aligned to each other.

The column index can also be an iterable of integers:
%cont-doctest
\begin{minted}{pycon}
>>> alignment[:, -3:]  # doctest:+ELLIPSIS
<Alignment object (2 rows x 3 columns) at ...>
>>> print(alignment[:, (1, 3, 0)])
target            0 ACG 3
                  0 --| 3
query             0 --G 1
<BLANKLINE>
\end{minted}

\section{Reverse-complementing the alignment}

Reverse-complementing an alignment will take the reverse complement of each sequence, and recalculate the coordinates:
%cont-doctest
\begin{minted}{pycon}
>>> print(alignment.sequences)
['GAACT', 'GAT']
>>> rc_alignment = alignment.reverse_complement()
>>> print(rc_alignment.sequences)
['AGTTC', 'ATC']
>>> print(rc_alignment)
target            0 AGTTC 5
                  0 |-|-| 5
query             0 A-T-C 3
<BLANKLINE>
>>> print(alignment[:, :4].sequences)
['GAACT', 'GAT']
>>> print(alignment[:, :4])
target            0 GAAC 4
                  0 |-|- 4
query             0 G-A- 2
<BLANKLINE>
>>> rc_alignment = alignment[:, :4].reverse_complement()
>>> print(rc_alignment[:, :4].sequences)
['AGTTC', 'ATC']
>>> print(rc_alignment[:, :4])
target            1 GTTC 5
                  0 -|-| 4
query             1 -T-C 3
<BLANKLINE>
\end{minted}
Reverse-complementing an alignment preserves its column annotations (in reverse order), but discards all other annotations.

\section{Exporting alignments}

Use the \verb+format+ method to create a string representation of the alignment in various file formats. This method takes an argument \verb+fmt+ specifying the file format, and may take additional keyword arguments depending on file type. The following values for \verb+fmt+ are supported:

\begin{itemize}
\item \verb+""+ (empty string; default): Create a human-readable representation of the alignment (same as when you \verb+print+ the alignment).
\item \verb+"SAM"+: Create a line representing the alignment in the Sequence Alignment/Map (SAM) format:
%cont-doctest
\begin{minted}{pycon}
>>> alignment.format("sam")
'query\t0\ttarget\t1\t255\t1M1D1M1D1M\t*\t0\t0\tGAT\t*\tAS:i:3\n'
\end{minted}
\item \verb+"BED"+: Create a line representing the alignment in the Browser Extensible Data (BED) file format:
%cont-doctest
\begin{minted}{pycon}
>>> alignment.format("bed")
'target\t0\t5\tquery\t3.0\t+\t0\t5\t0\t3\t1,1,1,\t0,2,4,\n'
\end{minted}
\item \verb+"PSL"+: Create a line representing the alignment in the Pattern Space Layout (PSL) file format as generated by BLAT \cite{kent2002}).
%cont-doctest
\begin{minted}{pycon}
>>> alignment.format("psl")
'3\t0\t0\t0\t0\t0\t2\t2\t+\tquery\t3\t0\t3\ttarget\t5\t0\t5\t3\t1,1,1,\t0,1,2,\t0,2,4,\n'
\end{minted}
The first four columns in the PSL output contain the number of matched and mismatched characters, the number of matches to repeat regions, and the number of matches to unknown nucleotides.
Repeat regions in the target sequence are indicated by masking the sequence as lower-case or upper-case characters, as defined by the following values for the \verb+mask+ keyword argument:
\begin{itemize}
\item \verb+False+ (default): Do not count matches to masked sequences separately;
\item \verb+"lower"+: Count and report matches to lower-case characters as matches to repeat regions;
\item \verb+"upper"+: Count and report matches to upper-case characters as matches to repeat regions;
\end{itemize}
The character used for unknown nucleotides is defined by the \verb+wildcard+ argument. For consistency with BLAT, the wildcard character is \verb+"N"+ by default. Use \verb+wildcard=None+ if you don't want to count matches to any unknown nucleotides separately.
%doctest
\begin{minted}{pycon}
>>> from Bio import Align
>>> aligner = Align.PairwiseAligner()
>>> aligner.mismatch_score = -1
>>> aligner.internal_gap_score = -5
>>> aligner.wildcard = "N"
>>> target = "AAAAAAAggggGGNGAAAAA"
>>> query = "GGTGGGGG"
>>> alignments = aligner.align(target.upper(), query)
>>> print(len(alignments))
1
>>> alignment = alignments[0]
>>> print(alignment)
target            0 AAAAAAAGGGGGGNGAAAAA 20
                  0 -------||.|||.|----- 20
query             0 -------GGTGGGGG-----  8
<BLANKLINE>
>>> alignment.score
5.0
>>> alignment.target
'AAAAAAAGGGGGGNGAAAAA'
>>> alignment.target = target
>>> alignment.target
'AAAAAAAggggGGNGAAAAA'
>>> print(alignment)
target            0 AAAAAAAggggGGNGAAAAA 20
                  0 -------....||.|----- 20
query             0 -------GGTGGGGG-----  8
<BLANKLINE>
>>> print(alignment.format("psl"))  # doctest: +NORMALIZE_WHITESPACE
6   1   0   1   0   0   0   0   +   query   8   0   8   target   20   7   15   1   8,   0,   7,
>>> print(alignment.format("psl", mask="lower"))  # doctest: +NORMALIZE_WHITESPACE
3   1   3   1   0   0   0   0   +   query   8   0   8   target   20   7   15   1   8,   0,   7,
>>> print(
...     alignment.format("psl", mask="lower", wildcard=None)
... )  # doctest: +NORMALIZE_WHITESPACE
3   2   3   0   0   0   0   0   +   query   8   0   8   target   20   7   15   1   8,   0,   7,
\end{minted}
\end{itemize}

In addition to the \verb+format+ method, you can use Python's built-in \verb+format+ function:
%cont-doctest
\begin{minted}{pycon}
>>> print(format(alignment, "psl"))  # doctest: +NORMALIZE_WHITESPACE
6   1   0   1   0   0   0   0   +   query   8   0   8   target   20   7   15   1   8,   0,   7,
\end{minted}
allowing \verb+Alignment+ objects to be used in formatted (f-) strings in Python:
\begin{minted}{pycon}
>>> print(
...     f"The alignment in PSL format is '{alignment:psl}'."
... )  # doctest: +NORMALIZE_WHITESPACE
The alignment in PSL format is '6   1   0   1   0   0   0   0   +   query   8   0   8   target   20   7   15   1   8,   0,   7,
'
\end{minted}
Note that optional keyword arguments cannot be used with the \verb+format+ function or with formatted strings.

\section{Reading and writing alignments}

\subsection{ClustalW}
\label{subsec:align_clustal}

\section{Substitution matrices}
\label{sec:substitution_matrices}

Substitution matrices \cite{durbin1998} provide the scoring terms for classifying how likely two different residues are to substitute for each other. This is essential in doing sequence comparisons.  Biopython provides a ton of common substitution matrices, including the famous PAM and BLOSUM series of matrices, and also provides functionality for creating your own substitution matrices.

\subsection{Array objects}

You can think of substitutions matrices as 2D arrays in which the indices are letters (nucleotides or amino acids) rather than integers.  The \verb+Array+ class in \verb+Bio.Align.substitution_matrices+ is a subclass of numpy arrays that supports indexing both by integers and by specific strings. An \verb+Array+ instance can either be a one-dimensional array or a square two-dimensional arrays. A one-dimensional \verb+Array+ object can for example be used to store the nucleotide frequency of a DNA sequence, while a two-dimensional \verb+Array+ object can be used to represent a scoring matrix for sequence alignments.

To create a one-dimensional \verb+Array+, only the alphabet of allowed letters needs to be specified:

%doctest . lib:numpy
\begin{minted}{pycon}
>>> from Bio.Align.substitution_matrices import Array
>>> counts = Array("ACGT")
>>> print(counts)
A 0.0
C 0.0
G 0.0
T 0.0
<BLANKLINE>
\end{minted}
The allowed letters are stored in the \verb+alphabet+ property:

%cont-doctest
\begin{minted}{pycon}
>>> counts.alphabet
'ACGT'
\end{minted}
This property is read-only; modifying the underlying \verb+_alphabet+ attribute may lead to unexpected results.
Elements can be accessed both by letter and by integer index:

%cont-doctest
\begin{minted}{pycon}
>>> counts["C"] = -3
>>> counts[2] = 7
>>> print(counts)
A  0.0
C -3.0
G  7.0
T  0.0
<BLANKLINE>
>>> counts[1]
-3.0
\end{minted}

Using a letter that is not in the alphabet, or an index that is out of bounds, will cause a \verb+IndexError+:

%cont-doctest
\begin{minted}{pycon}
>>> counts["U"]
Traceback (most recent call last):
    ...
IndexError: 'U'
>>> counts["X"] = 6
Traceback (most recent call last):
    ...
IndexError: 'X'
>>> counts[7]
Traceback (most recent call last):
    ...
IndexError: index 7 is out of bounds for axis 0 with size 4
\end{minted}

A two-dimensional \verb+Array+ can be created by specifying \verb+dims=2+:

%doctest . lib:numpy
\begin{minted}{pycon}
>>> from Bio.Align.substitution_matrices import Array
>>> counts = Array("ACGT", dims=2)
>>> print(counts)
    A   C   G   T
A 0.0 0.0 0.0 0.0
C 0.0 0.0 0.0 0.0
G 0.0 0.0 0.0 0.0
T 0.0 0.0 0.0 0.0
<BLANKLINE>
\end{minted}
Again, both letters and integers can be used for indexing, and specifying a letter that is not in the alphabet will cause an \verb+IndexError+:

%cont-doctest
\begin{minted}{pycon}
>>> counts["A", "C"] = 12.0
>>> counts[2, 1] = 5.0
>>> counts[3, "T"] = -2
>>> print(counts)
    A    C   G    T
A 0.0 12.0 0.0  0.0
C 0.0  0.0 0.0  0.0
G 0.0  5.0 0.0  0.0
T 0.0  0.0 0.0 -2.0
<BLANKLINE>
>>> counts["X", 1]
Traceback (most recent call last):
    ...
IndexError: 'X'
>>> counts["A", 5]
Traceback (most recent call last):
    ...
IndexError: index 5 is out of bounds for axis 1 with size 4
\end{minted}
Selecting a row or column from the two-dimensional array will return a one-dimensional \verb+Array+:

%cont-doctest
\begin{minted}{pycon}
>>> counts = Array("ACGT", dims=2)
>>> counts["A", "C"] = 12.0
>>> counts[2, 1] = 5.0
>>> counts[3, "T"] = -2
\end{minted}
% don't include this in the doctest, as the exact output is platform-dependent
\begin{minted}{pycon}
>>> counts["G"]
Array([0., 5., 0., 0.],
      alphabet='ACGT')
>>> counts[:, "C"]
Array([12.,  0.,  5.,  0.],
      alphabet='ACGT')
\end{minted}

\verb+Array+ objects can thus be used as an array and as a dictionary. They can be converted to plain numpy arrays or plain dictionary objects:

%cont-doctest
\begin{minted}{pycon}
>>> import numpy as np
>>> x = Array("ACGT")
>>> x["C"] = 5
\end{minted}
% don't include this in the doctest, as the exact output is platform-dependent
\begin{minted}{pycon}
>>> x
Array([0., 5., 0., 0.],
      alphabet='ACGT')
>>> a = np.array(x)  # create a plain numpy array
>>> a
array([0., 5., 0., 0.])
>>> d = dict(x)  # create a plain dictionary
>>> d
{'A': 0.0, 'C': 5.0, 'G': 0.0, 'T': 0.0}
\end{minted}

While the alphabet of an \verb+Array+ is usually a string, you may also use a tuple of (immutable) objects. This is used for example for a \hyperlink{codonmatrix}{codon substitution matrix}, where the keys are not individual nucleotides or amino acids but instead three-nucleotide codons.

While the \verb+alphabet+ property of an \verb+Array+ is immutable, you can create a new \verb+Array+ object by selecting the letters you are interested in from the alphabet. For example,
%cont-doctest
\begin{minted}{pycon}
>>> a = Array("ABCD", dims=2, data=np.arange(16).reshape(4, 4))
>>> print(a)
     A    B    C    D
A  0.0  1.0  2.0  3.0
B  4.0  5.0  6.0  7.0
C  8.0  9.0 10.0 11.0
D 12.0 13.0 14.0 15.0
<BLANKLINE>
>>> b = a.select("CAD")
>>> print(b)
     C    A    D
C 10.0  8.0 11.0
A  2.0  0.0  3.0
D 14.0 12.0 15.0
<BLANKLINE>
\end{minted}
Note that this also allows you to reorder the alphabet.

Data for letters that are not found in the alphabet are set to zero:
%cont-doctest
\begin{minted}{pycon}
>>> c = a.select("DEC")
>>> print(c)
     D   E    C
D 15.0 0.0 14.0
E  0.0 0.0  0.0
C 11.0 0.0 10.0
<BLANKLINE>
\end{minted}

As the \verb+Array+ class is a subclass of numpy array, it can be used as such. A \verb+ValueError+ is triggered if the \verb+Array+ objects appearing in a mathematical operation have different alphabets, for example

%doctest . lib:numpy
\begin{minted}{pycon}
>>> from Bio.Align.substitution_matrices import Array
>>> d = Array("ACGT")
>>> r = Array("ACGU")
>>> d + r
Traceback (most recent call last):
    ...
ValueError: alphabets are inconsistent
\end{minted}

\subsection{Calculating a substitution matrix from a pairwise sequence alignment}

As \verb+Array+ is a subclass of a numpy array, you can apply mathematical operations on an \verb+Array+ object in much the same way. Here, we illustrate this by calculating a scoring matrix from the alignment of the 16S ribosomal RNA gene sequences of \textit{Escherichia coli} and \textit{Bacillus subtilis}. First, we create a \verb+PairwiseAligner+ object (see Chapter~\ref{chapter:pairwise}) and initialize it with the default scores used by \verb+blastn+:

%doctest ../Tests/Align lib:numpy
\begin{minted}{pycon}
>>> from Bio.Align import PairwiseAligner
>>> aligner = PairwiseAligner(scoring="blastn")
>>> aligner.mode = "local"
\end{minted}
Next, we read in the 16S ribosomal RNA gene sequence of \textit{Escherichia coli} and \textit{Bacillus subtilis} (provided in \verb+Tests/Align/ecoli.fa+ and \verb+Tests/Align/bsubtilis.fa+), and align them to each other:

%cont-doctest
\begin{minted}{pycon}
>>> from Bio import SeqIO
>>> sequence1 = SeqIO.read("ecoli.fa", "fasta")
>>> sequence2 = SeqIO.read("bsubtilis.fa", "fasta")
>>> alignments = aligner.align(sequence1, sequence2)
\end{minted}
The number of alignments generated is very large:

%cont-doctest
\begin{minted}{pycon}
>>> len(alignments)
1990656
\end{minted}
However, as they only differ trivially from each other, we arbitrarily choose the first alignment, and count the number of each substitution:

%cont-doctest
\begin{minted}{pycon}
>>> alignment = alignments[0]
>>> substitutions = alignment.substitutions
>>> print(substitutions)
      A     C     G     T
A 307.0  19.0  34.0  19.0
C  15.0 280.0  25.0  29.0
G  34.0  24.0 401.0  20.0
T  24.0  36.0  20.0 228.0
<BLANKLINE>
\end{minted}
We normalize against the total number to find the probability of each substitution, and create a symmetric matrix of observed frequencies:

%cont-doctest
\begin{minted}{pycon}
>>> observed_frequencies = substitutions / substitutions.sum()
>>> observed_frequencies = (observed_frequencies + observed_frequencies.transpose()) / 2.0
>>> print(format(observed_frequencies, "%.4f"))
       A      C      G      T
A 0.2026 0.0112 0.0224 0.0142
C 0.0112 0.1848 0.0162 0.0215
G 0.0224 0.0162 0.2647 0.0132
T 0.0142 0.0215 0.0132 0.1505
<BLANKLINE>
\end{minted}
The background probability is the probability of finding an A, C, G, or T nucleotide in each sequence separately. This can be calculated as the sum of each row or column:

%cont-doctest
\begin{minted}{pycon}
>>> background = observed_frequencies.sum(0)
>>> print(format(background, "%.4f"))
A 0.2505
C 0.2337
G 0.3165
T 0.1993
<BLANKLINE>
\end{minted}
The number of substitutions expected at random is simply the product of the background distribution with itself:

%cont-doctest
\begin{minted}{pycon}
>>> expected_frequencies = background[:, None].dot(background[None, :])
>>> print(format(expected_frequencies, "%.4f"))
       A      C      G      T
A 0.0627 0.0585 0.0793 0.0499
C 0.0585 0.0546 0.0740 0.0466
G 0.0793 0.0740 0.1002 0.0631
T 0.0499 0.0466 0.0631 0.0397
<BLANKLINE>
\end{minted}
The scoring matrix can then be calculated as the logarithm of the odds-ratio of the observed and the expected probabilities:

%cont-doctest
\begin{minted}{pycon}
>>> oddsratios = observed_frequencies / expected_frequencies
>>> import numpy as np
>>> scoring_matrix = np.log2(oddsratios)
>>> print(scoring_matrix)
     A    C    G    T
A  1.7 -2.4 -1.8 -1.8
C -2.4  1.8 -2.2 -1.1
G -1.8 -2.2  1.4 -2.3
T -1.8 -1.1 -2.3  1.9
<BLANKLINE>
\end{minted}
The matrix can be used to set the substitution matrix for the pairwise aligner (see Chapter~\ref{chapter:pairwise}):

%cont-doctest
\begin{minted}{pycon}
>>> aligner.substitution_matrix = scoring_matrix
\end{minted}

\subsection{Calculating a substitution matrix from a multiple sequence alignment}
\label{sec:subs_mat_ex}

In this example, we'll first read a protein sequence alignment from the Clustalw file \href{examples/protein.aln}{protein.aln} (also available online
\href{https://raw.githubusercontent.com/biopython/biopython/master/Tests/Clustalw/protein.aln}{here})

%doctest ../Tests/Clustalw
\begin{minted}{pycon}
>>> from Bio import Align
>>> filename = "protein.aln"
>>> alignment = Align.read(filename, "clustal")
\end{minted}

Section~\ref{subsec:align_clustal} contains more information on doing this.

The \verb+substitutions+ property of the alignment stores the number of times
different residues substitute for each other:
%cont-doctest
\begin{minted}{pycon}
>>> substitutions = alignment.substitutions
\end{minted}

To make the example more readable, we'll select only amino acids with polar charged side chains:

%cont-doctest
\begin{minted}{pycon}
>>> substitutions = substitutions.select("DEHKR")
>>> print(substitutions)
       D      E      H      K      R
D 2360.0  270.0   15.0    1.0   48.0
E  241.0 3305.0   15.0   45.0    2.0
H    0.0   18.0 1235.0    8.0    0.0
K    0.0    9.0   24.0 3218.0  130.0
R    2.0    2.0   17.0  103.0 2079.0
<BLANKLINE>
\end{minted}
Rows and columns for other amino acids were removed from the matrix.

Next, we normalize the matrix and make it symmetric.
%cont-doctest
\begin{minted}{pycon}
>>> observed_frequencies = substitutions / substitutions.sum()
>>> observed_frequencies = (observed_frequencies + observed_frequencies.transpose()) / 2.0
>>> print(format(observed_frequencies, "%.4f"))
       D      E      H      K      R
D 0.1795 0.0194 0.0006 0.0000 0.0019
E 0.0194 0.2514 0.0013 0.0021 0.0002
H 0.0006 0.0013 0.0939 0.0012 0.0006
K 0.0000 0.0021 0.0012 0.2448 0.0089
R 0.0019 0.0002 0.0006 0.0089 0.1581
<BLANKLINE>
\end{minted}

Summing over rows or columns gives the relative frequency of occurrence of
each residue:
%cont-doctest
\begin{minted}{pycon}
>>> background = observed_frequencies.sum(0)
>>> print(format(background, "%.4f"))
D 0.2015
E 0.2743
H 0.0976
K 0.2569
R 0.1697
<BLANKLINE>
>>> background.sum()
1.0
\end{minted}

The expected frequency of residue pairs is then
%cont-doctest
\begin{minted}{pycon}
>>> expected_frequencies = background[:, None].dot(background[None, :])
>>> print(format(expected_frequencies, "%.4f"))
       D      E      H      K      R
D 0.0406 0.0553 0.0197 0.0518 0.0342
E 0.0553 0.0752 0.0268 0.0705 0.0465
H 0.0197 0.0268 0.0095 0.0251 0.0166
K 0.0518 0.0705 0.0251 0.0660 0.0436
R 0.0342 0.0465 0.0166 0.0436 0.0288
<BLANKLINE>
\end{minted}
Here, \verb+background[:, None]+ creates a 2D array consisting of a single column with the values of \verb+expected_frequencies+, and \verb+rxpected_frequencies[None, :]+ a 2D array with these values as a single row. Taking their dot product (inner product) creates a matrix of expected frequencies where each entry consists of two \verb+expected_frequencies+ values multiplied with each other. For example, \verb+expected_frequencies['D', 'E']+ is equal to \verb+residue_frequencies['D'] * residue_frequencies['E']+.

We can now calculate the log-odds matrix by dividing the observed frequencies by the expected frequencies and taking the logarithm:
%cont-doctest
\begin{minted}{pycon}
>>> import numpy as np
>>> scoring_matrix = np.log2(observed_frequencies / expected_frequencies)
>>> print(scoring_matrix)
      D    E    H     K    R
D   2.1 -1.5 -5.1 -10.4 -4.2
E  -1.5  1.7 -4.4  -5.1 -8.3
H  -5.1 -4.4  3.3  -4.4 -4.7
K -10.4 -5.1 -4.4   1.9 -2.3
R  -4.2 -8.3 -4.7  -2.3  2.5
<BLANKLINE>
\end{minted}

This matrix can be used as the substitution matrix when performing alignments. For example,
%cont-doctest
\begin{minted}{pycon}
>>> from Bio.Align import PairwiseAligner
>>> aligner = PairwiseAligner()
>>> aligner.substitution_matrix = scoring_matrix
>>> aligner.gap_score = -3.0
>>> alignments = aligner.align("DEHEK", "DHHKK")
>>> print(alignments[0])
target            0 DEHEK 5
                  0 |.|.| 5
query             0 DHHKK 5
<BLANKLINE>
>>> print("%.2f" % alignments.score)
-2.18
>>> score = (
...     scoring_matrix["D", "D"]
...     + scoring_matrix["E", "H"]
...     + scoring_matrix["H", "H"]
...     + scoring_matrix["E", "K"]
...     + scoring_matrix["K", "K"]
... )
>>> print("%.2f" % score)
-2.18
\end{minted}
(see Chapter~\ref{chapter:pairwise} for details).

\subsection{Reading \texttt{Array} objects from file}

\verb+Bio.Align.substitution_matrices+ includes a parser to read one- and two-dimensional \verb+Array+ objects from file. One-dimensional arrays are represented by a simple two-column format, with the first column containing the key and the second column the corresponding value. For example, the file \verb+hg38.chrom.sizes+ (obtained from UCSC), available in the \verb+Tests/Align+ subdirectory of the Biopython distribution, contains the size in nucleotides of each chromosome in human genome assembly hg38:
\begin{minted}{text}
chr1    248956422
chr2    242193529
chr3    198295559
chr4    190214555
...
chrUn_KI270385v1    990
chrUn_KI270423v1    981
chrUn_KI270392v1    971
chrUn_KI270394v1    970
\end{minted}
To parse this file, use

%doctest ../Tests/Align lib:numpy
\begin{minted}{pycon}
>>> from Bio.Align import substitution_matrices
>>> with open("hg38.chrom.sizes") as handle:
...     table = substitution_matrices.read(handle)
...
>>> print(table)  # doctest: +ELLIPSIS
chr1 248956422.0
chr2 242193529.0
chr3 198295559.0
chr4 190214555.0
...
chrUn_KI270423v1       981.0
chrUn_KI270392v1       971.0
chrUn_KI270394v1       970.0
<BLANKLINE>
\end{minted}
Use \verb+dtype=int+ to read the values as integers:

%cont-doctest
\begin{minted}{pycon}
>>> with open("hg38.chrom.sizes") as handle:
...     table = substitution_matrices.read(handle, int)
...
>>> print(table)  # doctest: +ELLIPSIS
chr1 248956422
chr2 242193529
chr3 198295559
chr4 190214555
...
chrUn_KI270423v1       981
chrUn_KI270392v1       971
chrUn_KI270394v1       970
<BLANKLINE>
\end{minted}

For two-dimensional arrays, we follow the file format of substitution matrices provided by NCBI. For example, the BLOSUM62 matrix, which is the default substitution matrix for NCBI's protein-protein BLAST \cite{altschul1990} program \verb+blastp+, is stored as follows:
\begin{minted}{text}
#  Matrix made by matblas from blosum62.iij
#  * column uses minimum score
#  BLOSUM Clustered Scoring Matrix in 1/2 Bit Units
#  Blocks Database = /data/blocks_5.0/blocks.dat
#  Cluster Percentage: >= 62
#  Entropy =   0.6979, Expected =  -0.5209
   A  R  N  D  C  Q  E  G  H  I  L  K  M  F  P  S  T  W  Y  V  B  Z  X  *
A  4 -1 -2 -2  0 -1 -1  0 -2 -1 -1 -1 -1 -2 -1  1  0 -3 -2  0 -2 -1  0 -4
R -1  5  0 -2 -3  1  0 -2  0 -3 -2  2 -1 -3 -2 -1 -1 -3 -2 -3 -1  0 -1 -4
N -2  0  6  1 -3  0  0  0  1 -3 -3  0 -2 -3 -2  1  0 -4 -2 -3  3  0 -1 -4
D -2 -2  1  6 -3  0  2 -1 -1 -3 -4 -1 -3 -3 -1  0 -1 -4 -3 -3  4  1 -1 -4
C  0 -3 -3 -3  9 -3 -4 -3 -3 -1 -1 -3 -1 -2 -3 -1 -1 -2 -2 -1 -3 -3 -2 -4
Q -1  1  0  0 -3  5  2 -2  0 -3 -2  1  0 -3 -1  0 -1 -2 -1 -2  0  3 -1 -4
E -1  0  0  2 -4  2  5 -2  0 -3 -3  1 -2 -3 -1  0 -1 -3 -2 -2  1  4 -1 -4
G  0 -2  0 -1 -3 -2 -2  6 -2 -4 -4 -2 -3 -3 -2  0 -2 -2 -3 -3 -1 -2 -1 -4
H -2  0  1 -1 -3  0  0 -2  8 -3 -3 -1 -2 -1 -2 -1 -2 -2  2 -3  0  0 -1 -4
...
\end{minted}
This file is included in the Biopython distribution under \verb+Bio/Align/substitution_matrices/data+. To parse this file, use

%doctest ../Bio/Align/substitution_matrices/data lib:numpy
\begin{minted}{pycon}
>>> from Bio.Align import substitution_matrices
>>> with open("BLOSUM62") as handle:
...     matrix = substitution_matrices.read(handle)
...
>>> print(matrix.alphabet)
ARNDCQEGHILKMFPSTWYVBZX*
>>> print(matrix["A", "D"])
-2.0
\end{minted}
The header lines starting with \verb+#+ are stored in the attribute \verb+header+:

%cont-doctest
\begin{minted}{pycon}
>>> matrix.header[0]
'Matrix made by matblas from blosum62.iij'
\end{minted}
We can now use this matrix as the substitution matrix on an aligner object:

%cont-doctest
\begin{minted}{pycon}
>>> from Bio.Align import PairwiseAligner
>>> aligner = PairwiseAligner()
>>> aligner.substitution_matrix = matrix
\end{minted}
To save an Array object, create a string first:

%cont-doctest
\begin{minted}{pycon}
>>> text = str(matrix)
>>> print(text)  # doctest: +ELLIPSIS
#  Matrix made by matblas from blosum62.iij
#  * column uses minimum score
#  BLOSUM Clustered Scoring Matrix in 1/2 Bit Units
#  Blocks Database = /data/blocks_5.0/blocks.dat
#  Cluster Percentage: >= 62
#  Entropy =   0.6979, Expected =  -0.5209
     A    R    N    D    C    Q    E    G    H    I    L    K    M    F    P    S ...
A  4.0 -1.0 -2.0 -2.0  0.0 -1.0 -1.0  0.0 -2.0 -1.0 -1.0 -1.0 -1.0 -2.0 -1.0  1.0 ...
R -1.0  5.0  0.0 -2.0 -3.0  1.0  0.0 -2.0  0.0 -3.0 -2.0  2.0 -1.0 -3.0 -2.0 -1.0 ...
N -2.0  0.0  6.0  1.0 -3.0  0.0  0.0  0.0  1.0 -3.0 -3.0  0.0 -2.0 -3.0 -2.0  1.0 ...
D -2.0 -2.0  1.0  6.0 -3.0  0.0  2.0 -1.0 -1.0 -3.0 -4.0 -1.0 -3.0 -3.0 -1.0  0.0 ...
C  0.0 -3.0 -3.0 -3.0  9.0 -3.0 -4.0 -3.0 -3.0 -1.0 -1.0 -3.0 -1.0 -2.0 -3.0 -1.0 ...
...
\end{minted}
and write the \verb+text+ to a file.

\subsection{Loading predefined substitution matrices}

Biopython contains a large set of substitution matrices defined in the literature, including BLOSUM (Blocks Substitution Matrix) \cite{henikoff1992} and PAM (Point Accepted Mutation) matrices \cite{dayhoff1978}. These matrices are available as flat files in the \verb+Bio/Align/substitution_matrices/data+ directory, and can be loaded into Python using the \verb+load+ function in the \verb+substitution_matrices+ submodule. For example, the BLOSUM62 matrix can be loaded by running

%doctest . lib:numpy
\begin{minted}{pycon}
>>> from Bio.Align import substitution_matrices
>>> m = substitution_matrices.load("BLOSUM62")
\end{minted}
This substitution matrix has an alphabet consisting of the 20 amino acids used in the genetic code, the three ambiguous amino acids B (asparagine or aspartic acid), Z (glutamine or glutamic acid), and X (representing any amino acid), and the stop codon represented by an asterisk:

%cont-doctest
\begin{minted}{pycon}
>>> m.alphabet
'ARNDCQEGHILKMFPSTWYVBZX*'
\end{minted}

To get a full list of available substitution matrices, use \verb+load+ without an argument:

%cont-doctest
\begin{minted}{pycon}
>>> substitution_matrices.load()  # doctest: +ELLIPSIS
['BENNER22', 'BENNER6', 'BENNER74', 'BLASTN', 'BLASTP', 'BLOSUM45', 'BLOSUM50', ..., 'TRANS']
\end{minted}

\hypertarget{codonmatrix}{
Note that the substitution matrix provided by Schneider \textit{et al.} \cite{schneider2005} uses an alphabet consisting of three-nucleotide codons:
}

%cont-doctest
\begin{minted}{pycon}
>>> m = substitution_matrices.load("SCHNEIDER")
>>> m.alphabet  # doctest: +ELLIPSIS
('AAA', 'AAC', 'AAG', 'AAT', 'ACA', 'ACC', 'ACG', 'ACT', ..., 'TTG', 'TTT')
\end{minted}

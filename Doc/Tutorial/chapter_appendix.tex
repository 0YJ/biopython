\chapter{Appendix: Useful stuff about Python}
\label{chapter:appendix}

If you haven't spent a lot of time programming in Python, many
questions and problems that come up in using Biopython are often
related to Python itself. This section tries to present some ideas and
code that come up often (at least for us!) while using the Biopython
libraries. If you have any suggestions for useful pointers that could
go here, please contribute!

\section{What the heck is a handle?}
\label{sec:appendix-handles}

Handles are mentioned quite frequently throughout this documentation,
and are also fairly confusing (at least to me!). Basically, you can
think of a handle as being a ``wrapper'' around text information.

Handles provide (at least) two benefits over plain text information:

\begin{enumerate}
  \item They provide a standard way to deal with information stored in
  different ways. The text information can be in a file, or in a
  string stored in memory, or the output from a command line program,
  or at some remote website, but the handle provides a common way of
  dealing with information in all of these formats.

  \item They allow text information to be read incrementally, instead
  of all at once. This is really important when you are dealing with
  huge text files which would use up all of your memory if you had to
  load them all.
\end{enumerate}

Handles can deal with text information that is being read (e.~g.~reading
from a file) or written (e.~g.~writing information to a file). In the
case of a ``read'' handle, commonly used functions are \verb|read()|,
which reads the entire text information from the handle, and
\verb|readline()|, which reads information one line at a time. For
``write'' handles, the function \verb|write()| is regularly used.

The most common usage for handles is reading information from a file,
which is done using the built-in Python function \verb|open|. Here,
we handle to the file \texttt{m\_cold.fasta} which you can download
\href{https://raw.githubusercontent.com/biopython/biopython/master/Doc/examples/m\_cold.fasta}{here}
(or find included in the Biopython source code as \texttt{Doc/examples/m\_cold.fasta}).

\begin{minted}{pycon}
>>> handle = open("m_cold.fasta", "r")
>>> handle.readline()
">gi|8332116|gb|BE037100.1|BE037100 MP14H09 MP Mesembryanthemum ...\n"
\end{minted}

Handles are regularly used in Biopython for passing information to parsers.
For example, since Biopython 1.54 the main functions in \verb|Bio.SeqIO|
and \verb|Bio.AlignIO| have allowed you to use a filename instead of a
handle:

\begin{minted}{python}
from Bio import SeqIO
for record in SeqIO.parse("m_cold.fasta", "fasta"):
    print(record.id, len(record))
\end{minted}

On older versions of Biopython you had to use a handle, e.g.

\begin{minted}{python}
from Bio import SeqIO
handle = open("m_cold.fasta", "r")
for record in SeqIO.parse(handle, "fasta"):
    print(record.id, len(record))
handle.close()
\end{minted}

This pattern is still useful - for example suppose you have a gzip
compressed FASTA file you want to parse:

\begin{minted}{python}
import gzip
from Bio import SeqIO
handle = gzip.open("m_cold.fasta.gz", "rt")
for record in SeqIO.parse(handle, "fasta"):
    print(record.id, len(record))
handle.close()
\end{minted}

With our parsers for plain text files, under Python 3 it is
essential to use gzip in text mode.

See Section~\ref{sec:SeqIO_compressed} for more examples like this,
including reading bzip2 compressed files.

\subsection{Creating a handle from a string}

One useful thing is to be able to turn information contained in a
string into a handle. The following example shows how to do this using
\verb|cStringIO| from the Python standard library:

%doctest
\begin{minted}{pycon}
>>> my_info = "A string\n with multiple lines."
>>> print(my_info)
A string
 with multiple lines.
>>> from StringIO import StringIO
>>> my_info_handle = StringIO(my_info)
>>> first_line = my_info_handle.readline()
>>> print(first_line)
A string
<BLANKLINE>
>>> second_line = my_info_handle.readline()
>>> print(second_line)
 with multiple lines.
\end{minted}
